\chapter{Conclusiones}

% **************************** Define Graphics Path **************************
\graphicspath{{Chapter5/Figs/}}

Se realizó una investigación del estado del arte de las diferentes aplicaciones que se le puede dar al paradigma SDN, prestando especial atención a las diferentes maneras de implementar redes privadas virtuales. También se hizo una investigación profunda sobre las diferentes opciones de virtualización para las redes definidas por software.

Tomando como base al popular emulador Mininet, se construyó una herramienta que permite emular dispositivos que pueden actuar como switches OpenFlow y al mismo tiempo correr protocolos distribuidos legados como OSPF. Esta herramienta permite la virtualización de la arquitectura RAUFlow, pero también se la puede ver como un resultado valioso incluso afuera del contexto de RAUFlow, ya que hasta el momento de realizar este trabajo no se encontró ninguna herramienta que tenga esas capacidades. Se creó un repositorio público en Github \cite{gitp2015_44} con el código fuente del emulador con la intención de hacerlo disponible para la comunidad. También se produjo un manual de usuario para facilitar el uso futuro de la herramienta.

Se trabajó en una verificación funcional del entorno construido, que permitió detectar y corregir problemas, y validar el correcto funcionamiento del mismo con diversas topologias. Dicha verificación también permitió corregir dos defectos en el código fuente de la aplicación RAUFlow, y detectar algunos posibles problemas de escalabilidad de la arquitectura. También se implementó una mejora a RAUFlow, que permite eliminar la necesidad de agentes SNMP en cada RAUSwitch, reduciendo la complejidad y posiblemente aumentando el rendimiento de los mismos.

Usando la herramienta construida se realizó una serie de pruebas para estudiar la escalabilidad de RAUFlow. En primer lugar, mediante diferentes topologias de prueba se estudió el tiempo requerido para la creación de VPNs. Para profundizar el análisis, se estudió la distribución del tiempo de ejecución entre las principales tareas que componen dicho proceso de creación. En segundo lugar, se llevó a cabo una serie de pruebas que estudian el comportamiento de RAUFlow y los RAUSwitch cuando existen muchos servicios. Desde el punto de vista de los RAUSwitch, un estudio a fondo de la herramienta Open vSwitch determinó que la existencia de muchos servicios no afecta el rendimiento de los mismos. Desde el punto de vista del controlador, se verificó que no tiene problemas para mantener muchos servicios, y se estudió la evolución de su consumo de memoria.

Por último, este trabajo contribuyó a la realización de una publicación científica llamada "RAUflow: building Virtual Private Networks with MPLS and OpenFlow" \cite{rauflow}, la cual fue presentada recientemente en la conferencia ACM SIGCOMM Workshop on Fostering Latin-American Research in Data Communication Networks (LANCOMM 2016). La publicación también ha sido aceptada en formato poster en Spring School on Networks (SSN 2016) a llevarse a cabo en noviembre.

%mencionar SSN si corresponde

\section{Trabajo futuro}

Si bien en este trabajo se ha desarrollado un emulador completamente funcional, aún existe lugar para mejorarlo. Por ejemplo, la configuración de nuevas topologias se debe hacer mediante scripts en Python. Si bien la API para hacerlo es relativamente simple, la tarea puede ser tediosa si se trata de topologias grandes. Existe una propuesta de interfaz gráfica para Mininet llamada MiniEdit \cite{miniedit} que puede ser utilizada con el propósito de mejorar la usabilidad del entorno virtual. Es probable que sea necesario modificar MiniEdit para que funcione con el entorno construido, pero esto no debería ser una tarea muy compleja ya que el código fuente está disponible y bien documentado.

El entorno construido también puede ser mejorado para proporcionar experimentos más reales, o fieles a la realidad. Existe una extensión de Mininet llamada Mininet-HiFi \cite{mininet-hifi} que apunta a mejorar Mininet para proveer experimentos más realistas y reproducibles. Utiliza conceptos como límites de CPU, control de tráfico y aislamiento de recursos para intentar que el experimento que se lleve a cabo sea lo más fiel posible al mismo escenario pero con hardware real.

Se han hecho diversas pruebas sobre la arquitectura RAUFlow que muestran que posee una buena escalabilidad. Sin embargo, es necesario hacer más pruebas para poder asegurarlo. Desde el punto de vista de la escalabilidad en las topologias, en la sección 3.5 se detallaron algunos problemas funcionales que se sospecha pueden afectar un despliegue real de la arquitectura, y por ende se sugiere una investigación a fondo sobre los mismos. Desde el punto de vista de la escalabilidad en el uso, no se hicieron pruebas con modelos de tráfico que reflejen el volumen y características del tráfico que se espera en la red. Esto es un paso clave en la determinación de la escalabilidad de RAUFlow.

Sería interesante implementar nuevas funcionalidades sobre la arquitectura RAUFlow que incorporen nociones como Ingeniería de tráfico, Quality of Service y seguridad de la red.

Se ha mencionado que el controlador de RAUFlow almacena todos sus datos en memoria. Implementar un método de persistencia que no dependa de la memoria del sistema podría mejorar la escalabilidad y también la robustez, ya que el controlador se podría recuperar ante fallas.

También se discutió que mantener comunicación con demasiados switches OpenFlow puede ser un problema para el controlador, lo cual presentaría un serio problema de escalabilidad. La existencia de un único centro de cómputo para el control de la red también puede generar un efecto cuello de botella. Estos problemas pueden ser solucionados utilizando múltiples controladores, y estableciendo una relación de jerarquía entre ellos.
