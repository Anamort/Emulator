\chapter{Estado del arte}

% **************************** Define Graphics Path **************************
\graphicspath{{Chapter2/Figs/}}

\section{Software Defined Networking}
Arquitectura \\
Modelo reactivo vs proactivo \\
%https://www.necam.com/docs/?id=2709888a-ecfd-4157-8849-1d18144a6dda

%http://www.iaria.org/conferences2014/filesICNS14/InfoSys%202014%20Control%20Plane%20Scalability%20in%20SDN-%20%20v1.2.pdf

%http://arxiv.org/pdf/1408.6760.pdf

\section{OpenFlow}

%http://article.sciencepublishinggroup.com/pdf/10.11648.j.ajsea.20140306.12.pdf

%http://www.spirent.com/~/media/White%20Papers/Broadband/PAB/OpenFlow_Performance_Testing_WhitePaper.pdf

\section{Open vSwitch}
%http://openvswitch.org/pipermail/discuss/2012-February/006447.html

%http://openvswitch.org/support/dist-docs/ovs-vswitchd.8.txt

%http://benpfaff.org/papers/ovs.pdf

%http://networkheresy.com/2014/11/13/accelerating-open-vswitch-to-ludicrous-speed/

%http://bluesy.wang/notes/2014/09/14/flow-caching-for-high-entropy-packet-fields/

%http://yuba.stanford.edu/~nickm/papers/flow-caching.pdf

%http://wangcong.org/2012/10/20/an-overview-of-openvswitch-implementation/

%https://www.google.com/url?sa=t&rct=j&q=&esrc=s&source=web&cd=1&ved=0ahUKEwixpuyBieLLAhUCdR4KHVV7CKgQFggfMAA&url=https%3A%2F%2Fnsrc.org%2Fworkshops%2F2014%2Fnznog-sdn%2Fraw-attachment%2Fwiki%2FAgenda%2FOpenVSwitch.pdf&usg=AFQjCNFg9VULvEmHMXQAsuTOE6XLH6WbzQ&bvm=bv.117868183,d.dmo&cad=rja

%http://openvswitch.org/slides/OpenStack-131107.pdf

%http://openvswitch.org/slides/openvswitch.en-2.pdf

%https://www.google.com/url?sa=t&rct=j&q=&esrc=s&source=web&cd=3&ved=0ahUKEwjP95D9ieLLAhVIpB4KHUKeCzYQFggoMAI&url=http%3A%2F%2Fopenvswitch.org%2Fsupport%2Fovscon2014%2F17%2F1530-ovsconf2014_chadnorgan.pptx&usg=AFQjCNEmnrXDB0psj5oXvoAzVYr8F5sLtQ&cad=rja

%https://www.google.com/url?sa=t&rct=j&q=&esrc=s&source=web&cd=1&cad=rja&uact=8&ved=0ahUKEwjZvdvaiuLLAhWF1h4KHfYrCKoQFggiMAA&url=http%3A%2F%2Fopenvswitch.org%2Fsupport%2Fovscon2014%2F18%2F1600-ovs_perf.pptx&usg=AFQjCNGGTT0rHmENeZ2ZcoRfWNFJ__hBLw

\section{Red Privada Virtual (VPN)}

\section{Multiprotocol Label Switching (MPLS)}

\section{Simuladores de redes}
%https://hal-univ-tlse3.archives-ouvertes.fr/hal-00605383/document
\subsection{Cloonix}
\subsection{CORE}
\subsection{DOT} %si
\subsection{Estinet} %si
\subsection{fs-sdn} %si
\subsection{GNS3}
\subsection{IMUNES}
\subsection{LINE}
\subsection{Marionnet}
\subsection{Mininet} %si
\subsection{MLN}
\subsection{Netkit}
\subsection{NetSim}
\subsection{NS-3} %si
\subsection{OFNet} %si
\subsection{Omnet++}
\subsection{Opencontrail} %si
\subsection{OpenStack}
\subsection{OPNET}
\subsection{Psimulator2}
\subsection{Shadow}
\subsection{STS} %si
\subsection{Unified Networking Lab}
\subsection{VNX y VNUML}

%http://netgroup.uniroma2.it/twiki/pub/Oshi/WebHome/OSHI_tech_report.pdf

%http://tiny-tera.stanford.edu/~nickm/papers/p253.pdf

%\section{Aplicaciones de SDN}

Se estudió el estado del arte en lo que respecta a opciones de emulación o simulación para SDN. A continuación se detallan las principales herramientas analizadas al momento de hacer esta investigación.\\

\textbf{NS-3}\\
ns-3 fue descartado debido a que no ofrece soporte para Quagga ni OpenFlow 1.3 al momento de realizar esta investigacion.\\

\textbf{Estinet}\\
Estinet requiere licencias pagas, y se opto por elegir herramientas open source. Debido a la falta de documentación de libre acceso, no se sabe que tipo de capacidades ofrece.\\

\textbf{Mininet}\\
Mininet es un emulador de redes SDN que permite emular hosts, switches, controladores y enlaces. Utiliza virtualización basada en procesos para ejecutar múltiples instancias (hasta 4096) de hosts y switches en un unico kernel de sistema operativo. También utiliza una capacidad de Linux denominada \textit{network namespace} que permite crear "interfaces de red virtuales", y de esta manera dotar a los nodos con sus propias interfaces, tablas de ruteo y tablas ARP. Lo que en realidad hace Mininet es utilizar la arquitectura \textit{Linux container}, que tiene la capacidad de proveer virtualización completa, pero de un modo reducido ya que no requiere de todas sus capacidades. Mininet también utiliza \textit{virtual ethernet (veth)} para crear los enlaces virtuales entre los nodos.

Mencionar MaxiNet (está en las referencias)
Capaz mencionar Mininet-HiFi (buscar en el survey)
Capaz mencionar Mininet CE (buscar en el survey)
Capaz mencionar SDN Cloud DC (buscar en el survey)



\textbf{LXC}\\
La opción de crear nodos con Linux containers resuelve el problema de Quagga y OpenFlow 1.3, pero llevaría una gran cantidad de trabajo construir distintas topologias (sobre todo si son grandes), ya que casi todo debe ser configurado manualmente por el usuario. Es una opción similar a Mininet, solo que sin gozar de todas las facilidades que ofrece esta última.\\


\textbf{Máquinas virtuales}\\
Es una opción similar a LXC (Linux Containers), sólo que menos escalable.
\\