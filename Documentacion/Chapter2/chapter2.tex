\chapter{Estado del arte}

% **************************** Define Graphics Path **************************
\graphicspath{{Chapter2/Figs/}}

\section{Software Defined Networking}
Aspectos generales
%https://www.necam.com/docs/?id=2709888a-ecfd-4157-8849-1d18144a6dda

%http://www.iaria.org/conferences2014/filesICNS14/InfoSys%202014%20Control%20Plane%20Scalability%20in%20SDN-%20%20v1.2.pdf

%http://arxiv.org/pdf/1408.6760.pdf

\section{OpenFlow}
Aspectos generales
%http://article.sciencepublishinggroup.com/pdf/10.11648.j.ajsea.20140306.12.pdf

%http://www.spirent.com/~/media/White%20Papers/Broadband/PAB/OpenFlow_Performance_Testing_WhitePaper.pdf

\section{Open vSwitch}
%http://openvswitch.org/pipermail/discuss/2012-February/006447.html

%http://openvswitch.org/support/dist-docs/ovs-vswitchd.8.txt

%http://benpfaff.org/papers/ovs.pdf

%http://networkheresy.com/2014/11/13/accelerating-open-vswitch-to-ludicrous-speed/

%http://bluesy.wang/notes/2014/09/14/flow-caching-for-high-entropy-packet-fields/

%http://yuba.stanford.edu/~nickm/papers/flow-caching.pdf

%http://wangcong.org/2012/10/20/an-overview-of-openvswitch-implementation/

%https://www.google.com/url?sa=t&rct=j&q=&esrc=s&source=web&cd=1&ved=0ahUKEwixpuyBieLLAhUCdR4KHVV7CKgQFggfMAA&url=https%3A%2F%2Fnsrc.org%2Fworkshops%2F2014%2Fnznog-sdn%2Fraw-attachment%2Fwiki%2FAgenda%2FOpenVSwitch.pdf&usg=AFQjCNFg9VULvEmHMXQAsuTOE6XLH6WbzQ&bvm=bv.117868183,d.dmo&cad=rja

%http://openvswitch.org/slides/OpenStack-131107.pdf

%http://openvswitch.org/slides/openvswitch.en-2.pdf

%https://www.google.com/url?sa=t&rct=j&q=&esrc=s&source=web&cd=3&ved=0ahUKEwjP95D9ieLLAhVIpB4KHUKeCzYQFggoMAI&url=http%3A%2F%2Fopenvswitch.org%2Fsupport%2Fovscon2014%2F17%2F1530-ovsconf2014_chadnorgan.pptx&usg=AFQjCNEmnrXDB0psj5oXvoAzVYr8F5sLtQ&cad=rja

%https://www.google.com/url?sa=t&rct=j&q=&esrc=s&source=web&cd=1&cad=rja&uact=8&ved=0ahUKEwjZvdvaiuLLAhWF1h4KHfYrCKoQFggiMAA&url=http%3A%2F%2Fopenvswitch.org%2Fsupport%2Fovscon2014%2F18%2F1600-ovs_perf.pptx&usg=AFQjCNGGTT0rHmENeZ2ZcoRfWNFJ__hBLw

\section{Multiprotocol Label Switching (MPLS)}
Conceptos básicos

\section{Red Privada Virtual (VPN)}
Conceptos básicos

\section{Aplicaciones de SDN}
\subsection{Implementación de VPN}
%https://blog.surf.nl/en/coco-an-exploration-of-software-defined-networking/
%http://www.nec.com/en/global/techrep/journal/g13/n02/pdf/130212.pdf
%http://yuba.stanford.edu/~nickm/papers/ofc11-saurav-mpls.pdf
\subsection{Otras aplicaciones}
Monitorización de red
Load balancing / QoS
Prevención de DoS
Implementación de Servicios de red

\section{Simuladores y emuladores}
\subsection{DOT} %es distribuido, asi que no aplicaria mucho, pero esta bueno, asi que capaz que si
\subsection{Estinet} %si
\subsection{fs-sdn} %si
\subsection{Mininet} %si
Mininet es un emulador de redes SDN que permite emular hosts, switches, controladores y enlaces. Utiliza virtualización basada en procesos para ejecutar múltiples instancias (hasta 4096) de hosts y switches en un unico kernel de sistema operativo. También utiliza una capacidad de Linux denominada \textit{network namespace} que permite crear "interfaces de red virtuales", y de esta manera dotar a los nodos con sus propias interfaces, tablas de ruteo y tablas ARP. Lo que en realidad hace Mininet es utilizar la arquitectura \textit{Linux container}, que tiene la capacidad de proveer virtualización completa, pero de un modo reducido ya que no requiere de todas sus capacidades. Mininet también utiliza \textit{virtual ethernet (veth)} para crear los enlaces virtuales entre los nodos.

Mencionar MaxiNet (está en las referencias)
Capaz mencionar Mininet-HiFi (buscar en el survey) %http://tiny-tera.stanford.edu/~nickm/papers/p253.pdf
Capaz mencionar Mininet CE (buscar en el survey)
Capaz mencionar SDN Cloud DC (buscar en el survey)
\subsection{NS-3} %si
%The popular network simulator NS-3 also offers an OpenFlow simulation model [8]. However, this model in its current version does not model the OpenFlow controller as an external entity. Therefore, it is not possible to quantify the effects of the control channel or simulate multiple switches connected to a single OpenFlow controller.
%tambien lo de openflow 1.3
\subsection{OFNet} %si, aunque parece estar muy en desarrollo todavia

\subsection{Otras herramientas}
\begin{itemize}
	\item Flowsim
	\item STS
	\item Omnet++
	\item Opencontrail
	\item CORE
\end{itemize}
%\subsection{Flowsim} %es mas bien didactico asi que no
%\subsection{STS} %es orientado a troubleshooting asi que no
%\subsection{Omnet++} %puede ser, pero no se puede usar rauflow, asi que capaz que no
%\subsection{Opencontrail} %creo que no, es un controlador asi que pareceria  que no
%\subsection{CORE} %usa ovs, asi que se podria mencionar aca

%\section{Simuladores/emuladores no SDN}
%\subsection{Unified Networking Lab} %no
%\subsection{VNX y VNUML} %no
%\subsection{OpenStack} %no
%\subsection{OPNET} %no
%\subsection{Psimulator2} %no
%\subsection{Shadow} %no
%\subsection{MLN} %no
%\subsection{Netkit}
%\subsection{NetSim} %no
%\subsection{GNS3} %no
%\subsection{LINE} %no
%\subsection{Marionnet} %no
%\subsection{Cloonix} %no

\section{Herramientas para pruebas de estrés y benchmarking}
Existen múltiples herramientas de testing aplicables en el paradigma SDN. De ellas se puede destacar un grupo, que no tienen como objetivo verificar aspectos funcionales, sino que apuntan a algo mucho más específico: las pruebas de estrés y el benchmarking. Ayudan a los administradores de red e investigadores a conocer los niveles de rendimiento de los cuales sus dispositivos y redes son capaces. Asimismo, pueden ser elementos de investigación útiles en el contexto de la nueva Red Académica, ya que podrían utilizarse, por ejemplo, para validar ciertos aspectos de rendimiento de los RAUSwitch.

En esta sección se estudiarán algunos ejemplos de estas herramientas, agrupándolas en dos grupos, de acuerdo a la entidad que intentan probar: switches y controladores. Para mantener la relevancia con el contexto de este trabajo, se limitará a herramientas enfocadas a OpenFlow.

\subsection{Testing de switches}
El testing y benchmarking de switches OpenFlow tiene como objetivo analizar cuál es el nivel máximo de rendimiento que puede alcanzar un switch. Esto por lo general se logra simulando las condiciones de una red que está bajo mucha carga, y sometiendo al switch a dichas condiciones, monitoreando su comportamiento. A continuación se listan algunas herramientas que hacen esto:
\begin{itemize}
	\item \textbf{OFLOPS} \cite{oflops} (OpenFlow Operations Per Second) es un framework open source para el testing y benchmarking de switches OpenFlow, tanto físicos como virtuales. Está desarrollado en el lenguaje C, y utiliza librerías de manipulación de paquetes para emular un controlador OpenFlow y tráfico de uso. Fue diseñado con un enfoque multi-thread para aprovechar las arquitecturas multi-core y así aumentar la potencia de la plataforma. Consiste de cinco threads paralelos, cada uno cumpliendo una función específica: 1) generación de paquetes, 2) captura de paquetes, 3) administración del canal de control (mensajes OpenFlow), 4) administración de un canal SNMP para hacer consultas asíncronas, y 5) un manejador de tiempo. Todo esto lo ofrece mediante una API, permitiendo a los usuarios crear sus propios módulos para escribir pruebas que se adapten a su realidad.
	\item \textbf{Spirent OpenFlow Controller Emulation} \cite{spirent-controller-emulation} es una herramienta desarrollada por la empresa Spirent\footnote{http://spirent.com/} que, igual que OFLOPS, tiene como propósito el testing y benchmarking de switches OpenFlow. Puede emular un controlador OpenFlow, definir millones de flujos y aplicar patrones de tráfico a esos flujos, y de ese modo medir el rendimiento, disponibilidad, seguridad y escalabilidad del switch. Entre sus funcionalidades, se destaca que puede probar todos los aspectos de OpenFlow 1.3, y que puede trabajar con switches híbridos. Estos dos puntos lo hacen una valiosa herramienta para el testing de la arquitectura RAUFlow.
\end{itemize}

\subsection{Testing de controladores}
Similar al caso de los switches, las pruebas de estrés sobre controladores OpenFlow consisten en someterlos a condiciones de mucha carga y estudiar determinadas métricas de su comportamiento. En general los principales aspectos que se buscan estudiar son 1) la cantidad de sesiones paralelas con switches OpenFlow que puede mantener el controlador y 2) el ritmo de mensajes packet\_in que puede manejar. No sólo es útil conocer esos umbrales, sino que también es muy valioso saber cual es el comportamiento esperado si se exceden dichos umbrales. Al ser RAUFlow un controlador de estilo proactivo, el aspecto número dos no es relevante aquí, ya que la red no genera paquetes de tipo packet\_in. A continuación se listan algunas de las herramientas disponibles:
\begin{itemize}
	\item \textbf{Cbench} \cite{cbench} es una herramienta para benchmarking de controladores, y es parte del proyecto OFLOPS. Su funcionamiento es muy simple: el usuario indica una cantidad \textit{n} de switches, la herramienta crea \textit{n} sesiones OpenFlow paralelas con el controlador, y luego comienza a enviar mensajes de tipo packet\_in y mide el tiempo que demora el controlador en responder a esos mensajes.
	\item \textbf{Spirent OpenFlow Switch Emulation} \cite{spirent-switch-emulation} es la propuesta de Spirent para el stress-testing de controladores OpenFlow. Igual que Cbench, en esencia consiste en emular múltiples switches y generar mensajes de tipo packet\_in. Sin embargo, es una solución un poco más sofisticada que Cbench, ya que permite trabajar con múltiples topologias y protocolos como ARP y LLDP.
\end{itemize}


