% ************************** Thesis Abstract *****************************
% Use `abstract' as an option in the document class to print only the titlepage and the abstract.
\begin{abstract2}
Las Redes Definidas por Software (SDN) es un paradigma de red que plantea desacoplar el plano de datos y el plano de control, además de estandarizar el modo en que se debe manipular a los dispositivos de red. Esto significa una gran mejora con respecto a las redes tradicionales, ya que permite a los administradores e investigadores desarrollar nuevos servicios y protocolos de forma más ágil y rápida, sin tener que preocuparse por aspectos comerciales de tecnologías propietarias ni de la complejidad de los protocolos distribuidos. Sin embargo, no es realista plantear un cambio repentino de un modelo a otro. Por lo tanto, es importante plantear trabajos que puedan aprovechar SDN y al mismo tiempo puedan convivir, e incluso aprovechar las tecnologías legadas. Un ejemplo de esto es RAUFlow, una propuesta de implementación desarrollada en 2015 para la nueva Red Académica Uruguaya. RAUFlow utiliza el enfoque SDN y al mismo tiempo ciertos aspectos de los protocolos legados para proveer redes privadas virtuales como servicio.

En este proyecto se continúa el trabajo sobre la arquitectura RAUFlow, teniendo como principal enfoque su escalabilidad, de cara a un posible despliegue en la nueva Red Académica Uruguaya. Para ello se construye una herramienta de virtualización que permite emular la arquitectura RAUFlow, y se utiliza la misma para hacer pruebas que no serían posibles con un prototipo físico. Además de ofrecer una herramienta con capacidades funcionales que no se han observado en ninguna otra herramienta disponible, se contribuye al campo con un estudio a fondo de las aplicaciones del modelo SDN, así como de sus tecnologías de virtualización disponibles.

\textbf{Palabras clave:} Redes de Computadoras, SDN, OpenFlow, Open vSwitch, Emuladores de red, Mininet
\end{abstract2}
