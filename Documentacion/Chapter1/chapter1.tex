\chapter{Introducción}
La Red Académica Uruguaya (RAU) es un emprendimiento de la Universidad de la República que ya lleva 28 años, y es administrado por el Servicio Central de Informática Universitario (SeCIU). Al igual que las redes académicas de muchos otros países, tiene como objetivo proveer a docentes e investigadores una infraestructura aislada de las redes comerciales, que les ofrezca altas velocidades para diversas aplicaciones de enseñanza e investigación. La RAU actualmente consta de 64 nodos y brinda servicios a 31 instituciones, abarcando un total aproximado de 8.000 docentes, 4.000 técnicos y 100.000 estudiantes\footnote{Cifras obtenidas del informe del proyecto RRAP}.

En la actualidad se está trabajando en renovar la RAU con una nueva infraestructura que permita proveer más y mejores servicios. Este esfuerzo se denominó RAU2. Los principales objetivos de esta renovación son: 1) extender y mejorar la calidad del servicio, y 2) implementar servicios e infraestructuras que maximicen la sinergia entre las instituciones. Para esto, se plantean tres áreas principales de trabajo: re-diseñar la topología de la red y aumentar el ancho de banda, mejorar la controlabilidad de la red, y desarrollar una plataforma integrada y global para los servicios académicos.

En 2015 culminó un proyecto llamado Routers Reconfigurables de Altas Prestaciones (RRAP) \cite{proyecto-rrap} que consistió en la construcción de un prototipo para la RAU2. Este prototipo propone una arquitectura de red denominada RAUFlow, y está basada en un concepto llamado Redes Definidas por Software (SDN por sus siglas en inglés). SDN es un paradigma de red que ha tomado fuerza en los últimos años, que plantea desacoplar el plano de control (donde se encuentra la inteligencia de la red) del plano de datos (donde se toman las acciones sobre los paquetes). Propone una infraestructura enfocada en otorgar, de forma más económica y eficiente, un mayor control a los operadores de red para adaptar y optimizar sus redes para los servicios y capacidades que necesitan proveer.

A pesar de la fuerza que han ganado las redes definidas por software en el último tiempo, están lejos de ser el estándar, y las redes legadas aún son muy utilizadas en la actualidad. Por esta razón es necesario estudiar soluciones que aprovechen las ventajas que ofrece SDN, y al mismo tiempo puedan coexistir con las redes legadas y beneficiarse de las mismas. RAUFlow es una propuesta que sigue ese mismo principio de redes híbridas, y es la base sobre la que se construye este trabajo.
% **************************** Define Graphics Path **************************
\graphicspath{{Chapter1/Figs/}}

\section{Proyecto RRAP}
El prototipo para la RAU2 construido por el proyecto RRAP \cite{proyecto-rrap} está compuesto 4 routers y una arquitectura para el control de la red basada en SDN, llamada RAUFlow. Cada dispositivo se denomina RAUSwitch y tiene la capacidad de funcionar en modo SDN y en modo tradicional como router IP. Los aspectos técnicos, tanto de RAUFlow como de RAUSwitch, se analizarán más adelante.
El prototipo fue verificado con una serie de pruebas funcionales que comprueban la validez del enfoque y que se proveen correctamente los servicios de red que fueron implementados.
Sin embargo, debido a las limitaciones físicas del prototipo, el enfoque propuesto aún no ha sido validado con pruebas de escala que aseguren que el mismo podría ser adoptado para una infraestructura con una dimensión y carga como la que tiene la RAU.

\section{Objetivos}
El objetivo principal de este trabajo es estudiar la escalabilidad de la arquitectura RAUFlow mediante simulaciones con grandes cantidades de nodos. Dada la ausencia de trabajos relacionados a la virtualización de redes híbridas que soporten SDN y los protocolos distribuidos tradicionales, previamente es necesario un estudio a fondo de las tecnologías de virtualización disponibles y un posterior trabajo de construcción de la plataforma deseada.

\section{Resultados esperados}
Se espera que este trabajo produzca los siguientes resultados:
\begin{itemize}
	\item El estado del arte en lo que refiere a la implementación de aplicaciones en SDN, así como las herramientas de virtualización disponibles, prestando especial atención a aquellas que pueden ser aplicables al enfoque híbrido SDN/Legacy y al mismo tiempo sean escalables.
	\item Una herramienta que permita virtualizar la arquitectura RAUFlow, y que sea razonablemente escalable. Es importante considerar dicha herramienta como un resultado que pueda mantenerse útil incluso afuera del contexto de RAUFlow, como herramienta de investigación autocontenida.
	\item Diseñar y llevar a cabo una serie de pruebas de escala sobre RAUFlow que detecten posibles errores, y permitan hacer un análisis sobre su escalabilidad.
\end{itemize}

\section{Estructura del documento}
La estructura de lo que resta de este documento se explica a continuación. En el capítulo 2 se muestran los resultados de la investigación del estado del arte. En el mismo se explican algunos conceptos claves para entender este trabajo, entre ellos los aspectos técnicos de RAUFlow y RAUSwitch. Luego se estudia lo investigado con respecto a las aplicaciones de SDN y las herramientas de virtualización disponibles. En el capítulo 3 se presentan los requerimientos, diseño, implementación y validación funcional del entorno virtual construido. En el capítulo 4 se presentan las pruebas de escala realizadas y un análisis de los resultados que arrojan. Por último, en el capítulo 5 se presentan las conclusiones de la realización de este trabajo y posibles líneas de trabajo futuro.

Además, se agrega un capítulo con la bibliografía utilizada y un apéndice con el manual de usuario para el entorno virtual.

