\section{Entorno virtual}
Dado que en el momento de la creación del prototipo no se contaba con un entorno de emulación, existía una fuerte limitación sobre la cantidad y calidad de pruebas que se podían realizar sobre la aplicación.

\subsection{Requerimientos del entorno virtual}
En primer lugar, el objetivo es que el entorno virtual se comporte de una forma lo más cercana posible al prototipo físico. Esto no quiere decir que deba usar las mismas herramientas, pero es deseable que así sea. A estos requerimientos, se agregan los requisitos inherentes de un entorno de virtualización como el que se pretende. A continuación se detallan los principales requerimientos a tener en cuenta.
\begin{itemize} 
	\item OSPF. El prototipo es dependiente en que los routers puedan utilizar este protocolo de enrutamiento, entonces resulta crucial que los nodos virtuales también puedan.
	\item OpenFlow 1.3: La aplicación que implementa VPNs sobre el prototipo depende de que los switches OpenFlow tengan soporte para MPLS, y OpenFlow ofrece MPLS a partir de la versión 1.3 (???).
	\item 
	\item Facilidad de configurar: Es importante que el entorno pueda generar distintas topologias y escenarios sin demasiado esfuerzo de configuración.
	\item Escalabilidad: Dado que uno de los objetivos de las pruebas es realizar pruebas de carga, el entorno debería ofrecer buena escalabilidad. Esto se traduce a que una computadora promedio de uso personal pueda levantar una decena de nodos virtuales como mínimo.
\end{itemize}

\subsection{Herramientas}
Se estudió el estado del arte en lo que respecta a opciones de emulación o simulación para SDN. A continuación se detallan las principales herramientas analizadas al momento de hacer esta investigación.\\

\textbf{NS-3}\\
ns-3 fue descartado debido a que no ofrece soporte para Quagga ni OpenFlow 1.3 al momento de realizar esta investigacion.\\

\textbf{Estinet}\\
Estinet requiere licencias pagas, y se opto por elegir herramientas open source. Debido a la falta de documentación de libre acceso, no se sabe que tipo de capacidades ofrece.\\

\textbf{Mininet}\\
Mininet es un emulador de redes SDN que permite emular hosts, switches, controladores y enlaces. Utiliza virtualización basada en procesos para ejecutar múltiples instancias (hasta 4096) de hosts y switches en un unico kernel de sistema operativo. También utiliza una capacidad de Linux denominada \textit{network namespace} que permite crear "interfaces de red virtuales", y de esta manera dotar a los nodos con sus propias interfaces, tablas de ruteo y tablas ARP. Lo que en realidad hace Mininet es utilizar la arquitectura \textit{Linux container}, que tiene la capacidad de proveer virtualización completa, pero de un modo reducido ya que no requiere de todas sus capacidades. Mininet también utiliza \textit{virtual ethernet (veth)} para crear los enlaces virtuales entre los nodos.


\textbf{LXC}\\
La opción de crear nodos con Linux containers resuelve el problema de Quagga y OpenFlow 1.3, pero llevaría una gran cantidad de trabajo construir distintas topologias (sobre todo si son grandes), ya que casi todo debe ser configurado manualmente por el usuario. Es una opción similar a Mininet, solo que sin gozar de todas las facilidades que ofrece esta última.\\


\textbf{Máquinas virtuales}\\
Es una opción similar a LXC (Linux Containers), sólo que menos escalable.
\\

\subsection{Diseño e implementación del entorno}
La herramienta que se eligió fue Mininet. \textit{Out of the box}, Mininet ya cumple tres de los cuatro requerimientos explicados anteriormente. Está diseñada para ser escalable, ya que usa containers reducidos, tiene soporte para OpenFlow 1.3 mediante OpenVSwitch, y es muy fácil de usar. El aspecto en el que falla es en el soporte para Quagga. Dado que Mininet es una herramienta de prototipado para SDN puro, no está pensado para un esquema híbrido como el que se propone. Los switches compatibles con OpenVSwitch que ofrece no pueden tener su propio network namespace, por lo tanto, no pueden tener su propia tabla de ruteo ni interfaces de red aisladas, así que no es posible que utilicen Quagga.

Por otro lado, los hosts de Mininet sí tienen su propio network namespace, y gracias a su capacidad de aislar procesos, podemos ejecutar una instancia de Quagga y OpenVSwitch para cada host. De esta forma logramos un router como el requerido por la arquitectura. Esta extensión de las funcionalidades de los hosts es posible ya que Mininet está programado con orientación a objetos y permite al usuario crear subclases propias de sus clases.

Insertar imagen de las clases****

Insertar imagen de arquitectura de archivos de Mininet****